%%
% TUM Corporate Design LaTeX Templates
% Based on the templates from https://www.tum.de/cd
%
% Feel free to join development on
% https://gitlab.lrz.de/tum-templates/templates
% and/or to create an issue in case of trouble.
%
% tum-article class for scientific articles, reports, exercise sheets, ...
%
%%

\documentclass[twocolumn]{tum-article}
%\documentclass[twocolumn, german]{tum-article}
%\documentclass[times, twocolumn]{tum-article}
%\documentclass[times]{tum-article}
%\documentclass{tum-article}

\usepackage{lipsum}

\title{Modifying the Writing Style in Goal-Oriented Dialog Generation}
\author{Fabian Bell\authormark{1,\Letter}\orcid{0000-0001-9595-4226},
  Monika Wintergerst\authormark{1}\orcid{0000-0002-9244-5431}}

% if too long for running head
\titlerunning{TUM Article}
\authorrunning{Author 1 et al.}

\email{fabian.bell@tum.de}

\affil[1]{Department of Informatics, Technical University of Munich (TUM),
  Boltzmannstr. 3, 85748 Garching, Germany}

\date{Received: 10 August 2017 / Accepted: 02 Februar 2018\thanks{This is a
    post-peer-review, pre-copyedit version of an article published in Fancy
    Journal. The final authenticated version is available online at:
    \url{http://dx.doi.org/}}}

\begin{document}

\maketitle

\begin{abstract}
  Abstract goes here. \lipsum[1]
\end{abstract}

\section{Introduction}

Dialogue systems (e.g. Chat-bots) have made great progress during the last years, mainly because of the invention of novel neural models and machine learning techniques. Because of the progress, current research aims to improve the dialogue systems on an individual and human-like level. Recent articles \cite[]{DBLP:journals/corr/abs-1901-08149, liu2020impress} tried to tackle this problem by feeding a persona into the network. The network is then trained to extract characteristic information from the given persona in order to generate a human-like output sequence. The persona is defined by a set of profile sentences like \textit{I like books}. The problem of these approaches consist of two tasks:
\begin{itemize}
\item \textbf{Persona understanding:}\\
The network needs to understand what the characteristics of the given persona are. This also yields the question on what defines a certain persona and on how to formulate the profile sentences, since the Persona definition has to be done manually. 
\item \textbf{Style derivation:}\\
The network needs to derive a certain speaking/writing style from the given persona. Profile sentences like \textit{I bought my first home} \cite[]{liu2020impress} do not necessarily contribute to a distinct writing style.
\end{itemize}


\section{Theory}

\lipsum[3-4]

\section{Experimental Setup}

\lipsum[4-5]

\section{Results}

\lipsum[6]

\section{Conclusions}

\lipsum[7]

\section*{Acknowledgements}

\lipsum[8]

\bibliographystyle{IEEEtran}
\bibliography{literature}

\end{document}
